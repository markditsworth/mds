\documentclass{amsart}

\usepackage{amsmath}
\usepackage{amsfonts}
\usepackage{amssymb}
\usepackage{graphicx}

\title{Problem Set 2}
\author{Mark Ditsworth}

\begin{document}
	\maketitle
	\section{Problem 1}
	Given graph $G=(V,E,W)$ consider a random walk on V with transition probabilities $M_{ij} = P{X(t+1)=j | X(t)=i} = \frac{w_{ij}}{\text{deg}(i)}$.
	\\\\
	Partition the vertex set as $V = V_+ \cup V_- \cup V_*$. Suppose that every node in $V_*$ is connected to at least one node in either $V_+$ or $V_-$. Given a node $i \subset V$ let $g(i)$ be the probability that a random walker starting at $i$ reaches a node in $V_+$ before reaching one in $V_-$. If $i \in V_+$, then $g(i)=1$ and if $i \in V_-$, then $g(i)=0$. Find $g(i)$ for $i \in V_*$.
	\\\\
	
	There are two scenarios in which the random walker will reach a node in $V_+$ before one in $V_-$: Moving from $V_*$ to $V_+$ immediately, or moving from $V_*$ to $V_*$ repeatedly until moving to $V_+$. We essentially view leaving $V_*$ as an absorbing barrier, and want the probability that we end in $V_+$, which is the sum of all probabilities that will end in that scenario:
	\[
	\sum_{j \in V_+}M_{ij} + \sum_{j \in V_*}\sum_{k \in V_+}M_{ij}M_{jk} + \sum_{j \in V_*}\sum_{k \in V_*}\sum_{l \in V_+}M_{ij}M_{jk}M_{kl} + \dots
	\]
	Define $\mathbf{\nu_*} \in \mathbb{R}^n$, $n=|V|$ where the $j$th element is $1$ if $j \in V_*$, and $0$ otherwise. Similarly define $\mathbf{\nu_+} \in \mathbb{R}^n$ for the partition $V_+$. Let $\Psi \in \mathbb{R}^n$, $n=|V|$ have elements everywhere equal 0 except for the $i$th node that is the starting node. The probability that a random walker starting at node $i$ is absorbed in the $V_+$ partition at time-step $k$ is expressed as
	\[
	\Psi^T \left[ M\text{diag}\left(V_*\right) \right]^{(k-1)} MV_+
	\]
	The $\left[ M\text{diag}\left(V_*\right) \right]^{(k-1)}$ matrix expresses the probability of starting at node $i$, and ending at node $j$ after $k-1$ steps, discounting the intermediate nodes that would absorb the random walker earlier (nodes belonging to $V_+$ or $V_-$) For convenience, we will denote this matrix as $M_*$, as in only accounting for transitions within partition $V_*$. The resulting matrix is dotted with $M$ for the last transition time step, and then dotted with $V_+$ to attain the vector of probabilities end in partition $V_+$. $\Psi^T$ dotted with this vector selects the probability stemming from starting at node $i$. Thus, summing the probabilities across all $k$ yields
	\[
	\Psi^T \left[\sum_{k=0}^{\infty}\left( M_* \right)^k\right]MV_+
	\]
	With each element of $M_*$ less than $0$, and the summation across each row bounded above by $1$, it is clear that $\lim\limits_{n \rightarrow \infty} (M_*)^n = \mathbf{0}$. Thus, the infinite sum converges to $(I-M_*)^{-1}$. In conclusion, the probability that a random walker starting at node $i \in V_*$ will reach a node in $V_+$ before a node in $V_-$ is calculated by
	\[
	g(i) = \Psi^T\left( I - M_* \right)^{-1}MV_+.
	\]
	Note that for graphs with large $|V|$, the inverse operation can be quite expensive. Thus $g(i)$ should be approximated either via pseudo-inverse operations or monte carlo simulations.
	\\\\
	\section{Problem 2}
	
	
\end{document}