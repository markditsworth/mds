\documentclass{amsart}

\usepackage{amsmath}
\usepackage{amsfonts}
\usepackage{amssymb}
\usepackage{graphicx}

\title{Problem Set 2}
\author{Mark Ditsworth}

\begin{document}
	\maketitle
	\section{Problem 2.1}
	Given graph $G=(V,E,W)$ consider a random walk on V with transition probabilities $M_{ij} = P{X(t+1)=j | X(t)=i} = \frac{w_{ij}}{\text{deg}(i)}$.
	\\\\
	Partition the vertex set as $V = V_+ \cup V_- \cup V_*$. Suppose that every node in $V_*$ is connected to at least one node in either $V_+$ or $V_-$. Given a node $i \subset V$ let $g(i)$ be the probability that a random walker starting at $i$ reaches a node in $V_+$ before reaching one in $V_-$. If $i \in V_+$, then $g(i)=1$ and if $i \in V_-$, then $g(i)=0$. Find $g(i)$ for $i \in V_*$.
	\\\\
	\textbf{Solution:}
	There are two scenarios in which the random walker will reach a node in $V_+$ before one in $V_-$: Moving from $V_*$ to $V_+$ immediately, or moving from $V_*$ to $V_*$ repeatedly until moving to $V_+$. We essentially view leaving $V_*$ as an absorbing barrier, and want the probability that we end in $V_+$, which is the sum of all probabilities that will end in that scenario:
	\[
	\sum_{j \in V_+}M_{ij} + \sum_{j \in V_*}\sum_{k \in V_+}M_{ij}M_{jk} + \sum_{j \in V_*}\sum_{k \in V_*}\sum_{l \in V_+}M_{ij}M_{jk}M_{kl} + \dots
	\]
	Define $\mathbf{\nu_*} \in \mathbb{R}^n$, $n=|V|$ where the $j$th element is $1$ if $j \in V_*$, and $0$ otherwise. Similarly define $\mathbf{\nu_+} \in \mathbb{R}^n$ for the partition $V_+$. Let $\Psi_i \in \mathbb{R}^n$, $n=|V|$ have elements everywhere equal 0 except for the $i$th node that is the starting node. The probability that a random walker starting at node $i$ is absorbed in the $V_+$ partition at time-step $k$ is expressed as
	\[
	\Psi_i^T \left[ M\text{diag}\left(V_*\right) \right]^{(k-1)} MV_+
	\]
	The $\left[ M\text{diag}\left(V_*\right) \right]^{(k-1)}$ matrix expresses the probability of starting at node $i$, and ending at node $j$ after $k-1$ steps, discounting the intermediate nodes that would absorb the random walker earlier (nodes belonging to $V_+$ or $V_-$) For convenience, we will denote this matrix as $M_*$, as in only accounting for transitions within partition $V_*$. The resulting matrix is dotted with $M$ for the last transition time step, and then dotted with $V_+$ to attain the vector of probabilities end in partition $V_+$. $\Psi_i^T$ dotted with this vector selects the probability stemming from starting at node $i$. Thus, summing the probabilities across all $k$ yields
	\[
	\Psi_i^T \left[\sum_{k=0}^{\infty}\left( M_* \right)^k\right]MV_+
	\]
	With each element of $M_*$ less than $0$, and the summation across each row bounded above by $1$, it is clear that $\lim\limits_{n \rightarrow \infty} (M_*)^n = \mathbf{0}$. Thus, the infinite sum converges to $(I-M_*)^{-1}$. In conclusion, the probability that a random walker starting at node $i \in V_*$ will reach a node in $V_+$ before a node in $V_-$ is calculated by
	\[
	g(i) = \Psi_i^T\left( I - M_* \right)^{-1}MV_+.
	\]
	Note that for graphs with large $|V|$, the inverse operation can be quite expensive. Thus $g(i)$ should be approximated either via pseudo-inverse operations or monte carlo simulations. See the attached notebook \textsf{ps2.ipynb} for justification of this closed-form expression for $g(i)$.
	\\\\
	\section{Problem 2.2}
	For a graph $G$ let $h(G)$ denote its Cheeger constant and $\lambda_2(\mathcal{L}_G)$ the second-smallest eigenvalue of its normalized graph Laplacian ($\mathcal{L}_G = D - W$). The Cheeger inequality guarantees that 
	\[
	\frac{1}{2}\lambda_2(\mathcal{L}_G) \le h_G \le \sqrt{2\lambda_2(\mathcal{L}_G)}
	\]
	This excercise shows that this inequality is tight (at least up to constants).
	
	1. Construct a family of graphs $\mathcal{G}$ for which $\lambda_2(\mathcal{L}_G) \rightarrow 0$ and for which there exists a constant $C>0$ for which 
	\[
	\forall G \in \mathcal{G}, h_G \le C\lambda_2(\mathcal{L}_G)
	\]
	\\\\
	
	2. Construct a family of graphs $\mathcal{G}$ for which $\lambda_2(\mathcal{L}_G) \rightarrow 0$ and for which there exists a constant $c>0$ for which 
	\[
	\forall G \in \mathcal{G}, h_G \ge c\sqrt{\lambda_2(\mathcal{L}_G)}
	\]
	\\\\
	
	\section{Problem  2.3}
	Given a graph $G$ show that the dimension of the nullspace of $\mathcal{L}_G$ corresponds to the number of connected components of $G$.
	\\\\
	\textbf{Solution:} The nullspace of $\mathcal{L}_G$ is the set of linearly independent vectors $\mathcal{X}$ such that $\forall \mathbf{x} \in \mathcal{X}$, $\mathcal{L}_G\mathbf{x} = \mathbf{0}$.
	By expanding $\mathcal{L}_G$ we can see that both $D$ and $W$ transform each $\mathbf{x}$ to equal vectors.
	\[
	\mathcal{L}_G \mathbf{x} = (D - W)\mathbf{x} = D\mathbf{x} - W\mathbf{x} = \mathbf{0}
	\]
	\[
	D\mathbf{x} = W\mathbf{x}
	\]
	Consider that graph $G = (V,E,W),~(|V|=n)$ has $k$ connected components. Let $\mathbf{x}^i$ be a vector where
	\[
	x^i_j = 
	\begin{cases}
		1 & j \in \text{ component } i\\
		0 & \text{otherwise} 
	\end{cases}
	\]
	Clearly, without loss of generality, $D\mathbf{x}^i$ is equal to a vector containing the degree of each node in $G$ that is in component $i$. $W\mathbf{x}^i$ is clearly the same since for each node, it will sum all weights of edges that connect to nodes in component $i$. Thus, there will be $k$ linearly independent vectors that are transformed to $\mathbf{0}$ by $\mathcal{L}_G$. All other vectors that do this will be linear combinations of $\mathcal{X} = \{\mathbf{x}^1,\dots ,\mathbf{x}^k\}$. Therefor, the cardinality of the nullspace of $\mathcal{L}_G$ is equal to the number of conneted components in graph $G$.
	\\\\
	\section{Problem 2.4}
	Given a connected unweighted graph $G = (V,E)$, its diameter is equal to 
	\[
	\text{diam}(G) = \max_{u,v\in V} \min_{path p, u\rightarrow v} \text{len}(p)
	\]
	Show that 
	\[
	\text{diam}(G) \ge \frac{1}{\text{vol}(G)\lambda_2(\mathcal{L}_G)}
	\]
	\\\\
	\section{Problem 2.5}
	Prove the Courant Fisher Theorem: Given a symmetric matrix $A\in \mathbb{R}^{n\times n}$ with eigenvalues $\lambda_1 \le \dots \lambda_n$, for $k \le n$,
	\[
	\lambda_k(A) = \min_{U:\text{dim}(U)=k} \left[ \max_{x\in U} \frac{x^TAx}{x^Tx} \right].
	\]
	\\
	Also show that
	\[
	\lambda_2(A) = \max_{y\in\mathbb{R}^n} \left[ \min_{x\in\mathbb{R}^n:x\perp y} \frac{x^TAx}{x^Tx} \right]
	\]
	\\\\
	\section{Problem 2.6}
	Given a set of points $x_1,\dots,x_n \in \mathbb{R}^p$ and a partition of them in $k$ clusters $S_1,\dots,S_k$ reca the k-means objective
	\[
	\min_{S_1,\dots,S_k}\min_{\mu_1,\dots,\mu_k}\sum_{l=1}^{k}\sum_{i\in S_l}\| x_i - \mu_l \|^2.
	\]
	Show that this is equivalent to 
	\[
	\min_{S_1,\dots,S_k}\sum_{l=1}^{k}\frac{1}{S_l}\sum_{i,j\in S_l}\| x_i - x_j \|^2.
	\]
	\\
	\textbf{Solution:}
	\[
	\min_{S_1,\dots,S_k}\min_{\mu_1,\dots,\mu_k}\sum_{l=1}^{k}\sum_{i\in S_l}\| x_i - \mu_l \|^2 =
	\]
	\[
	\min_{S_1,\dots,S_k}\sum_{l=1}^{k}\sum_{i\in S_l}\|x_i - \frac{1}{|S_l|}\sum_{j\in S_l}x_j\|^2 =
	\]
	\[
	\min_{S_1,\dots,S_k}\sum_{l=1}^{k}\frac{1}{|S_l|}\sum_{i\in S_l}\| |S_l|x_i - \sum_{j\in S_l}x_j\|^2 =
	\]
	\[
	\min_{S_1,\dots,S_k}\sum_{l=1}^{k}\frac{1}{|S_l|}\sum_{i\in S_l}\sum_{j\in S_l}\| x_i - x_j\|^2 =
	\]
	\[
	\min_{S_1,\dots,S_k}\sum_{l=1}^{k}\frac{1}{|S_l|}\sum_{i,j\in S_l}\| x_i - x_j\|^2.
	\]
	
	
\end{document}